\documentclass[journal]{IEEEtran}
\usepackage[spanish]{babel}
\usepackage[utf8]{inputenc}
\usepackage{amsmath, amsthm, amsfonts}
\usepackage{graphicx,amssymb}
\usepackage{cite}
\usepackage{epsfig}
\usepackage[usenames]{color}
\usepackage{listings}
\usepackage{tabularx}
\usepackage{multirow, array}
\usepackage{colortbl}
\usepackage{xcolor}
\renewcommand{\refname}{Referencias}
\usepackage{subfig}
\usepackage{listings}
\usepackage{booktabs}
\usepackage{cite}
\usepackage{float}
\usepackage{longtable}
\definecolor{anti-flashwhite}{rgb}{0.95, 0.95, 0.96}
\renewcommand{\citedash}{ -- }

\lstset{
  backgroundcolor=\color{anti-flashwhite},
  frame=single,
  basicstyle=\footnotesize
}

\title{\LARGE \bf Taller de laboratorio No. 2:
CONFIGURACIONES CON DIODOS
}

\author{\IEEEauthorblockN{~Andres  Felipe Aponte Lopez, Juan Nicolas Carvajal Barón,		Jaime Andrés Sanchez Peralta~ \\ \IEEEmembership{aapontel@unal.edu.co} {jncarvajalb@unal.edu.co} {jaasanchezpe@unal.edu.co} ~}\\

\IEEEauthorblockA{Universidad Nacional de Colombia\\
Facultad de Ingeniería\\
Electrónica Análoga I\\
 Colombia}
}

\renewcommand{\leftmark}{Taller de laboratorio No. 3:
DIODO ZENER Y AMPLIFICADOR OPERACIONAL

, Mayo 2021} 

\begin{document}

\maketitle

\section{Marco teórico}

\subsection{Diodos Zener}
Son diodos que están diseñados para trabajar en la región
de ruptura en la relación entre tensión corriente. Su funcionamiento se basa en que se puede polarizar en inverso, y a diferencia de los otros diodos, este conduce corriente si se tiene un nivel de tensión que es un valor casi constante.

\subsection{Características diodo 1N4733}

\begin{itemize}
    \item $V_z$ = 5.1 V
    \item $I_z$ = 49 mA
    \item $I_{ZK}$ = 1 mA
    \item $I_{Zmax}$ = 178 mA
    \item $r_{z@I_{z}}$ = 7 $\Omega$
    \item $r_{z@I_{zk}}$ = 550 $\Omega$
    \item $P_{max}$ = 1 W
\end{itemize}

\textbf{Modelo Spice.}
.model D1N4733 D(Is=1.214f Rs=1.078 Ikf=0 N=1 Xti=3
Eg=1.11 + Cjo=185p M=.3509 Vj=.75 Fc=.5 Isr=2.601n
Nr=2 Bv=5.1 Ibv=.70507 + Nbv=.74348 Ibvl=4.8274m
Nbvl=6.7393 Tbv1=176.471u) * Motorola pid=1N4733
case=DO-41 * 89-9-19 gjg * Vz = 5.1 @ 49mA, Zz = 175
@1mA, Zz = 8.2 @5mA, Zz = 2.2 @20mA

\subsection{Características diodo 1N4733}

\begin{itemize}
    \item $V_z$ = 8.2 V
    \item $I_z$ = 31 mA
    \item $I_{ZK}$ = 0.5 mA
    \item $I_{Zmax}$ = 110 mA
    \item $r_{z@I_{z}}$ = 4.5 $\Omega$
    \item $r_{z@I_{zk}}$ = 700 $\Omega$
    \item $P_{max}$ = 1 W
\end{itemize}


\textbf{Modelo Spice.}

.model D1N4738 D(Is=2.102f Rs=2.5 Ikf=0 N=1 Xti=3
Eg=1.11 Cjo=100p M=.3503 + Vj=.75 Fc=.5 Isr=2.252n
Nr=2 Bv=8.2 Ibv=8 Nbv=.53621 + Ibvl=213.52u
Nbvl=.17879 Tbv1=585.37u) * Motorola pid=1N4738
case=DO-41 * 89-9-19 gjg * Vz = 8.2 @ 31mA, Zz = 16
@ 1mA, Zz = 6.9 @ 5mA, Zz = 2.5 @ 20mA

\subsection{Amplificador operacional}
Son dispositivos de ganancia de energıa, diseñados con
el fin de proporcionar la función de transferencia deseada.
Este compuesto por dos entradas y una salida, donde la
salida es la diferencia de las dos entradas multiplicado por
una ganancia G: $V_{out}$ = G * (V+ - V-).


\section{Cálculos}

\subsection{Circuito 1}
Asumiendo que $V_1 = V_z + 10$, calcule el valor de R2
de tal manera que cuando $R_{1} = 0 \Omega$ corriente en el diodo
cumpla $I_z < I_{zmax}$; y el valor de $R_1$ de tal manera que
cuando $R_1 = R_{1max}$ la corriente en el diodo cumpla
$I_z \simeq I_{zmin}$




\section{Simulación}

\section{Montaje experimental}



\end{document}